\documentclass[]{article}
\usepackage{lmodern}
\usepackage{amssymb,amsmath}
\usepackage{ifxetex,ifluatex}
\usepackage{fixltx2e} % provides \textsubscript
\ifnum 0\ifxetex 1\fi\ifluatex 1\fi=0 % if pdftex
  \usepackage[T1]{fontenc}
  \usepackage[utf8]{inputenc}
\else % if luatex or xelatex
  \ifxetex
    \usepackage{mathspec}
  \else
    \usepackage{fontspec}
  \fi
  \defaultfontfeatures{Ligatures=TeX,Scale=MatchLowercase}
\fi
% use upquote if available, for straight quotes in verbatim environments
\IfFileExists{upquote.sty}{\usepackage{upquote}}{}
% use microtype if available
\IfFileExists{microtype.sty}{%
\usepackage{microtype}
\UseMicrotypeSet[protrusion]{basicmath} % disable protrusion for tt fonts
}{}
\usepackage[margin=1in]{geometry}
\usepackage{hyperref}
\hypersetup{unicode=true,
            pdfborder={0 0 0},
            breaklinks=true}
\urlstyle{same}  % don't use monospace font for urls
\usepackage{graphicx,grffile}
\makeatletter
\def\maxwidth{\ifdim\Gin@nat@width>\linewidth\linewidth\else\Gin@nat@width\fi}
\def\maxheight{\ifdim\Gin@nat@height>\textheight\textheight\else\Gin@nat@height\fi}
\makeatother
% Scale images if necessary, so that they will not overflow the page
% margins by default, and it is still possible to overwrite the defaults
% using explicit options in \includegraphics[width, height, ...]{}
\setkeys{Gin}{width=\maxwidth,height=\maxheight,keepaspectratio}
\IfFileExists{parskip.sty}{%
\usepackage{parskip}
}{% else
\setlength{\parindent}{0pt}
\setlength{\parskip}{6pt plus 2pt minus 1pt}
}
\setlength{\emergencystretch}{3em}  % prevent overfull lines
\providecommand{\tightlist}{%
  \setlength{\itemsep}{0pt}\setlength{\parskip}{0pt}}
\setcounter{secnumdepth}{0}
% Redefines (sub)paragraphs to behave more like sections
\ifx\paragraph\undefined\else
\let\oldparagraph\paragraph
\renewcommand{\paragraph}[1]{\oldparagraph{#1}\mbox{}}
\fi
\ifx\subparagraph\undefined\else
\let\oldsubparagraph\subparagraph
\renewcommand{\subparagraph}[1]{\oldsubparagraph{#1}\mbox{}}
\fi

%%% Use protect on footnotes to avoid problems with footnotes in titles
\let\rmarkdownfootnote\footnote%
\def\footnote{\protect\rmarkdownfootnote}

%%% Change title format to be more compact
\usepackage{titling}

% Create subtitle command for use in maketitle
\newcommand{\subtitle}[1]{
  \posttitle{
    \begin{center}\large#1\end{center}
    }
}

\setlength{\droptitle}{-2em}

  \title{}
    \pretitle{\vspace{\droptitle}}
  \posttitle{}
    \author{}
    \preauthor{}\postauthor{}
    \date{}
    \predate{}\postdate{}
  

\begin{document}

\section{Interpreting results: humility pays
off}\label{interpreting-results-humility-pays-off}

In everyone's research, programming mistakes (bugs) happen with higher
frequency than we would want. This is inevitable given that programming
is an inexact science -- and it takes long time to get good at it. Not
to mention putting together code that is partly yours and uses multiple
functions developed by others.

It becomes criticial then to do the following:

\textsc{Carefully inspect and critically evaluate your results.}

I've noticed that I often detect problems when they should have been
spotted by others. Sometimes the detection is rather easy, which still
surprises me. This leads me to think that learning to be a good
researcher depends on working hard on the statement of the previous
paragraph. Let me give a concrete example, though the example itself is
not important.

Suppose you're learning a new technique, something like network/graph
theory. There are a dozen basic measures (efficiency, modularity,
characteristic path length, etc.), and you're learning about them. Then
you compute these measures on a data sample, one measure per individual
(and for each condition). How are the measures similar or different? You
then compute the correlation between the measures across participants.

What would it mean for the correlation between two measures (say,
characteristic path length and efficiency for example) to be exactly 1?
How about 0? Two different measures cannot be correlated 100\% unless
they are mathematically and implementationally the exact same thing. In
a real sample, they also cannot be correlated 0\%, as sampling
variability and noise will always produce a correlation that is
different from zero, even if very small. So if you observed these values
it should be a very strong indication that something is wrong (with p
approaching 1\ldots{}).

\begin{figure}
\includegraphics[width=0.2\linewidth]{images/correlationmatrix1} \caption{Correlation matrix. The diagonals should be $1.0$ by definition, but can any off-diagonal assume $1.0$?}\label{fig:unnamed-chunk-1}
\end{figure}

What if you observed the correlation between two measures as \(0.99\)?
Could this happen?

\begin{figure}
\includegraphics[width=0.2\linewidth]{images/TOM_spearman_corr_g_eff_mean_connect_type2_N=100abs} \caption{Correlation scatter. The two measures are correlated $0.9983$.}\label{fig:unnamed-chunk-2}
\end{figure}

Logically speaking, it could. But if the two measures are distinct, even
if they were designed to capture the same information, this seems
very-highly unlikely. This is a good example because it doesn't imply an
implementational mistake somewhere but it really suggests that something
strange is going on. One should definitely track down what's going on
before moving on.

Here we're getting back to an earlier post on doing research without
ground truth. It calls for looking into the problem quite carefully.

So go back to the principle of ``doing research'' that started this
post:

\texttt{Carefully\ inspect\ and\ critically\ evaluate\ your\ results.}

And yes, do this every time: it pays off.


\end{document}
